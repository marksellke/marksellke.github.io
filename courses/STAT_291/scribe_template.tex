\documentclass[12pt]{memoir}

% Update lecture details
\newcommand{\lectureNum}{1}
\newcommand{\scribeName}{Alpha Beta}
\newcommand{\lectureDate}{1/1/1000}

\usepackage{kpfonts}
\usepackage{microtype}
\usepackage{hyperref}
\usepackage{graphicx}
\usepackage{geometry}
\usepackage{tikz-cd}
\usepackage{mathrsfs}
\newcommand{\etal}{{\itshape et al.}}
\newcommand{\proofEnd}{\hfill $\diamond$}
\newcommand{\Exp}{{\mathscr{E}}}
\newcommand{\Prb}{{\mathscr{P}}}

%--------------------------- Custom Layout and Settings -----------------
\usepackage[american]{babel}
\usepackage{amsfonts}
\usepackage{amsthm}
\usepackage{fancyhdr}
\renewcommand{\baselinestretch}{1.4}
\newfont{\headingFont}{cmssbx10 scaled 1300}

\newcounter{lectureItem} \setcounter{lectureItem}{0}
\newcommand{\lectureItem}[1]{\par \refstepcounter{lectureItem} {\headingFont \arabic{lectureItem}.~ #1} \par}

\definecolor{customNoteColor}{rgb}{0.2,0.4,0.6}
\newcommand{\customNote}[1]{{\color{customNoteColor}\textit{#1}}}
\newcommand{\increaseSpacing}{\small\renewcommand{\baselinestretch}{2.0}\normalsize}
\newcommand{\defaultSpacing}{\small\renewcommand{\baselinestretch}{1.4}\normalsize}
\newtheorem{theorem}{Theorem}[chapter]
\newtheorem{lemma}[theorem]{Lemma}
\newtheorem{proposition}[theorem]{Proposition}
\newtheorem{definition}{Definition}[chapter]
\newtheorem{remark}[definition]{Remark}
\newtheorem{example}[theorem]{Example}
\renewenvironment{proof}{{\bfseries Proof:}}{\proofEnd\par}
\newenvironment{proofOf}[1]{{\bfseries Proof of #1:}}{\proofEnd\par}
\newenvironment{proofSketch}{{\scshape Proof Sketch:}}{\proofEnd\par}

\pagestyle{fancy}
\fancyhf{}
\fancyhead[LE,RO]{\thepage}
\fancyhead[LO]{\nouppercase{\rightmark}}
\fancyhead[RE]{\nouppercase{\leftmark}}
\renewcommand{\headrulewidth}{0pt}

\hypersetup{
    colorlinks=true,
    linkcolor=blue,
    filecolor=magenta,      
    urlcolor=cyan,
}

\begin{document}
\begin{center}
\framebox{\parbox{6.5in}{
{\bfseries STAT 291, Spring 2024
\\
 Random High-Dimensional Optimization:
Landscapes and Algorithmic Barriers}
\href{https://canvas.harvard.edu/courses/126859}{Canvas}, \href{https://msellke.com/courses/STAT_291/course_page_website}{Course Page} .\\ Instructor: Mark Sellke\\\ \\
{\bfseries Lecture \lectureNum, \lectureDate. Notes prepared by \scribeName.}
}}
\ \\
\end{center}
\setcounter{chapter}{\lectureNum}
% Custom Note: ---------- Begin Lecture Notes Here ------------------------


This is a Latex template to scribe in-class lectures. 
Please write up the lecture content during class, and email your Latex and pdf files to Yufan and myself after class, by the next morning at the latest. Feel free to add your favorite macros to the Latex preamble.


If it is difficult to follow the lecture, scribing days are an especially good time to ask questions. If needed, you can also indicate points of confusion in your notes, which we can help resolve before posting.



Example unlabelled equation:
\[
    a+b=c.
\]
Example equation, labelled \eqref{eq:example-equation}:
\begin{equation}
\label{eq:example-equation}
    a^2+b^2=c^2.
\end{equation}
Example multi-line equation, labelled \eqref{eq:example-multi-line}:
\begin{equation}
\label{eq:example-multi-line}
\begin{aligned}
    a^3+b^3
    &=c^3,
    \\
    &\neq c^2.
\end{aligned}
\end{equation}




\end{document}
